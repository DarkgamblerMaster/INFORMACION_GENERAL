\section{Capítulo 1: Configuraciones}
De la siguiente manera se configura el usuario y el email en nuestro entorno git:

\begin{lstlisting}[numbers=none]
  git config --global user.name "username"
  git config --global user.email "email"
\end{lstlisting}

Para poder ver los valores ya configurados en el entorno se escribe el siguiente comando:

\begin{lstlisting}[numbers=none]
  git config --global -l
\end{lstlisting}

Para configurar el end of file se usa el siguiente comando

\begin{lstlisting}[numbers=none]
  git config --global core.autocrlf true
\end{lstlisting}

Para agregar credenciales al cache, primero se necesita hacer un push con lo que se solicitar\'a el username y el token como password, luego de que se hizo el push correctamente se debe escribir:


\begin{lstlisting}[numbers=none]
  git config --global credential.helper cache
\end{lstlisting}

si estamos en entorno windows entonces se debe escribir lo siguiente:


\begin{lstlisting}[numbers=none]
  git config --global credential.helper wincred
\end{lstlisting}

Con el siguiente c\'odigo se pueden agregar alias, que sirven para sustituir texto de comandos git con el fin de no escribir tanto, por ejemplo si queremos acortar commit -m con el nombre com:


\begin{lstlisting}[numbers=none]
  git config --global alias.com "commit -m"
\end{lstlisting}

Para borrar el alias que anteriormente creamos simplemente escribimos:

\begin{lstlisting}[numbers=none]
  git config --global --unset alias.com
\end{lstlisting}

Si se desea ver las configuraciones hechas en el git, as\'i como los alias y otras informaciones podemos abrir el archivo de configuraciones en emacs u otro

\begin{lstlisting}[numbers=none]
  emacs ~/.gitconfig
\end{lstlisting}

O bien lo podemos ver en la terminal:


\begin{lstlisting}[numbers=none]
  git config --global --list
\end{lstlisting}

\section{Cap\'itulo 2: Crear un repositorio}

Las siguientes son las instrucciones para crear un repositorio:


\begin{lstlisting}[numbers=none]
  echo "cualquier texto para el readme" >> README.md
  git init
  git add README.md
  git commit -m "first commit"
  git remote add origin <url_repositorio_github>
  git push -u origin master
\end{lstlisting}

git init se puede inicializar en un directorio agregando simplemente el nombre del directorio que se quiere crear


\begin{lstlisting}[numbers=none]
  git init <NOMBRE_dIRECTORIO>
\end{lstlisting}

Si se desea agregar todo lo que hay se puede usar el comando add de la siguiente manera


\begin{lstlisting}[numbers=none]
  git add .
\end{lstlisting}

Si se hizo alg\'un cambio en otra cuenta asociada al branch o en la p\'agina de github que queremos actualizar en local debemos escribir lo siguiente:


\begin{lstlisting}[numbers=none]
  git pull
\end{lstlisting}

Si queremos revertir el pull hecho entonces primero debemos encontrar el ID commit usando:
\begin{lstlisting}[numbers=none]
  git log
\end{lstlisting}

y luego:
\begin{lstlisting}[numbers=none]
  git reset --hard <commit id>
\end{lstlisting}

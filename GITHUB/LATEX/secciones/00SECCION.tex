\section{Capítulo 1: Configuraciones}
De la siguiente manera se configura el usuario y el email en nuestro entorno git:

\begin{lstlisting}[numbers=none]
  git config --global user.name "username"
  git config --global user.email "email"
\end{lstlisting}

Para poder ver los valores ya configurados en el entorno se escribe el siguiente comando:

\begin{lstlisting}[numbers=none]
  git config --global -l
\end{lstlisting}

Para configurar el end of file se usa el siguiente comando

\begin{lstlisting}[numbers=none]
  git config --global core.autocrlf true
\end{lstlisting}

Para agregar credenciales al cache, primero se necesita hacer un push con lo que se solicitar\'a el username y el token como password, luego de que se hizo el push correctamente se debe escribir:


\begin{lstlisting}[numbers=none]
  git config --global credential.helper cache
\end{lstlisting}

si estamos en entorno windows entonces se debe escribir lo siguiente:


\begin{lstlisting}[numbers=none]
  git config --global credential.helper wincred
\end{lstlisting}

Con el siguiente c\'odigo se pueden agregar alias, que sirven para sustituir texto de comandos git con el fin de no escribir tanto, por ejemplo si queremos acortar commit -m con el nombre com:


\begin{lstlisting}[numbers=none]
  git config --global alias.com "commit -m"
\end{lstlisting}

Para borrar el alias que anteriormente creamos simplemente escribimos:

\begin{lstlisting}[numbers=none]
  git config --global --unset alias.com
\end{lstlisting}

Si se desea ver las configuraciones hechas en el git, as\'i como los alias y otras informaciones podemos abrir el archivo de configuraciones en emacs u otro

\begin{lstlisting}[numbers=none]
  emacs ~/.gitconfig
\end{lstlisting}

O bien lo podemos ver en la terminal:


\begin{lstlisting}[numbers=none]
  git config --global --list
\end{lstlisting}
